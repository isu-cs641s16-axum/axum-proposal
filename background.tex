\section{Background}

\subsection{Pig-Latin Compilation}
\begin{frame}{Pig Latin Compilation}
\begin{itemize}
	\item Parser verifies the programs, performs type checking, schema inference
          and other tasks.
	\item It outputs a canonical logical plan.
	\item The logical plan is optimized and compiled to a series of MapReduce
          Jobs.
	\item The DAG of optimized Map-Reduce jobs are sorted topologically and jobs
          are submitted to Hadoop for execution.
\end{itemize}
\end{frame}

\subsection{MapReduce Execution Model}
\begin{frame}{MapReduce Execution Model}
\begin{itemize}
	\item Map: Produces a stream of data items annotated with keys.
	\item Local Sort: Orders the data at each machine by key.
	\item Combiner: Performs partial aggregation on the locally ordered data.
	\item Shuffle: Redistributes the data to achieve global ordering.
	\item Merge/Combine: All data received at a particular machine is combined
          into a single stream in merge stage.
	\item Reduce: Finally the reduce stage processes the data associated with
          each key and performs aggregation of the output results.
\end{itemize}
\end{frame}

\begin{frame}{MapReduce Execution Model}
\centerline{\includegraphics[scale=0.5]{Images/MapReduce_Execution.JPG}}
\let\thefootnote\relax\footnotetext{\tiny\citet[VLDB][]{gates2009building}}
\end{frame}

\subsection{Pig Latin's Logical Plan}
\begin{frame}{Logical Plan Structure}
\begin{itemize}
	\item A Pig Latin program is a sequence of steps, each of which carries out
          a single transformation.
	\item Each Pig Latin program is translated to a logical plan
	\item Pig then translates the logical plan to a physical plan and embeds
          each physical operator inside a MapReduce stage to arrive at the
          MapReduce plan.
\end{itemize}
\end{frame}

\begin{frame}{Logical Plan}
\centerline{\includegraphics[scale=0.55]{Images/PigLatin.JPG}}
\let\thefootnote\relax\footnotetext{\tiny \citet[VLDB][]{gates2009building}}
\end{frame}

\subsection{Generating MapReduce Jobs}
\begin{frame}{Logical-MapReduce}
\begin{itemize}
	\item Pig translates the logical plan to a physical plan
	\item Logical (CO)GROUP operator translates to - local rearrange, global
          rearrange and package.
	\item The JOIN is handled either with a COGROUP followed by a FOREACH or
          fragment replicate join.
	\item After the physical plan is generated Pig assigns physical operators to
          Hadoop Stages
\end{itemize}
\end{frame}

\begin{frame}
\centerline{\includegraphics[scale=0.40]{Images/Logical_Physical.JPG} }
\let\thefootnote\relax\footnotetext{\tiny \citet[VLDB][]{gates2009building}}
\end{frame}

\begin{frame}
\centerline{\includegraphics[scale=0.40]{Images/Physical_MapReduce.JPG}}
\let\thefootnote\relax\footnotetext{\tiny \citet[VLDB][]{gates2009building}}
\end{frame}

\begin{frame}{Pig to MapReduce}
\begin{itemize}
	\item Pig compiles programs written in Pig Latin
	\item Ultimately translates to Map-Reduce tasks
	\item Pig also operates on local mode running on a single machine without
          map-reduce
	\item There lies an equivalence between PigLatin and MapReduce operational
          semantics
	\item The equivalence needs to be formalized and correctness needs to be
          proved
\end{itemize}
\end{frame}


\subsection{Three Main Papers}

\begin{frame}{\citet[SEFM]{ono2011using}}
\emph{Using Coq in Specification and Program Extraction of Hadoop MapReduce
Applications}
\begin{itemize}
    \item Constructed formal model of MapReduce computation in Coq where:
    \begin{itemize}
        \item Model user-defined map/reduce as Coq functions
        \item Application specifications are defined as invariants
    \end{itemize}
	\item Constructed formal model of the (previously informally specified)
          Hadoop libraries
	\item Using these, they prove the correctness of some example MapReduce
          applications.
    \item Future Work: Investigate a more MapReduce applications
\end{itemize}
\end{frame}

\begin{frame}{\citet[ABZ][]{pereverzeva2014formal}}
\emph{Formal Derivation of Distributed MapReduce}
\begin{itemize}
	\item They proposed an approach to formal derivation, but it relies on
          stepwise refinement of Event-B.
	\item This approach also relies largely on proof based verification of
          Event-B and does not provide any scope for extension of mapreduce.
\end{itemize}
\end{frame}

\begin{frame}{\citet[VLDB][]{gates2009building}}
\emph{Building a HighLevel Dataflow System on top of MapReduce: The Pig
Experience}:
\begin{itemize}
	\item This paper outlines the compilation phase of PigLatin programs and how
          the logical plan of PigLatin is translated to actual Hadoop map-reduce
          tasks.
	\item But neither the semantics nor the compilation to map-reduce framework
          have been formalized with proof of correctness.
\end{itemize}
\end{frame}
